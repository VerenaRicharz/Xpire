% !TEX root =  master.tex
\section{Grundidee von Xpire}
In Deutschland landen jährlich fast 13 Millionen Tonnen Lebensmittel in der Mülltonne. Runter gerechnet ergibt das 85,2 Kilogramm Essen pro Jahr pro Haushalt. Zu diesen Ergebnissen kommt eine Studie aus dem Jahr 2005, durchgeführt von der Universität Stuttgart.\autocite[vgl.][]{BundesministeriumfurErnahrungLandwirtschaftundVerbraucherschutz.2012} Ein Teil dieser enormen Verschwendung ist zurückzuführen auf Landwirte, Lebensmittelverarbeiter, Handel und Gastronomie. Mehr als die Hälfte der Abfälle fällt jedoch in privaten Haushalten an. Eine Verbesserung dieser Situation ist nur erreichbar, wenn Verbraucher lernen weniger einzukaufen, Lebensmittel wie Obst, Gemüse und Brot richtig zu lagern und Reste nicht bedenkenlos wegzuwerfen.\\
Dank der Digitalisierung, die inzwischen fast alle Lebensbereiche durchdrungen hat und der Tatsache, dass weit mehr als die Hälfte aller Deutschen ein Smartphone besitzen (Tendenz steigend), können neue Technologien helfen, dieses Problem zu lösen. Wir, eine Gruppe von Studierenden der DHBW Mannheim, sehen daher ein großes Potential in der Entwicklung einer mobilen Applikation zur Verwaltung gekaufter Lebensmittel. Mit unserer mobilen Applikation \enquote{Xpire} wollen wir diesem Problem entgegen wirken und jedem auf einfache Art und Weise zu mehr Nachhaltigkeit verhelfen.
Die Grundidee besteht darin, Xpire als mobile Applikation möglichst simpel und benutzerfreundlich zu gestalten. Zu viele Funktionen und Features schrecken den Benutzer ab und führen schnell zu Verwirrung und Frustration. Die Hauptfunktionalität beschränkt sich daher auf:
\begin{itemize}[noitemsep]
	\item das Hinzufügen eines Produktes
	\item das Anzeigen aller hinterlegten Produkte mit Verfallsdatum
	\item das Benachrichtigen, falls die Haltbarkeit eines Produktes kurz vor dem ablaufen ist
\end{itemize}
Die App enthält auch weitere Funktionen wie das Hinzufügen von Produktbildern, das Löschen eines hinterlegten Produktes oder das Verändern der gespeicherten  Informationen zu einem Produkt.\\
Ziel ist es alle gekauften Lebensmittel problemlos über die App verwalten zu können und ortsunabhängig immer zu wissen, wie der Haltbarkeitsstand der einzelnen Produkte ist. So soll die App dabei helfen, Produkte innerhalb der Haltbarkeit zu verwerten und weniger Abfall zu produzieren. Dies schont nicht nur den Geldbeutel jedes Einzelnen, sondern trägt auch zu einer nachhaltigeren Lebensweise bei und hilft somit verantwortungsbewusster mit den endlichen Ressourcen unserer Erde umzugehen.

% Welche Arbeitsteilung haben wir genutzt und wieso?