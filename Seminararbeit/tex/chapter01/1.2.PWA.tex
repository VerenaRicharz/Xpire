% !TEX root =  master.tex
\section{Was ist eine PWA?}
Bei der traditionellen Entwicklung mobiler Anwendungen war die Wiederverwendbarkeit von Code zwischen nativen Anwendungen, dem Web und mobilen Plattformen von Natur aus nicht gegeben. Dies ist auf die nicht kompatible Codebasis nativer Anwendungen zurückzuführen, was zu getrennten Projekten und Entwicklerumgebungen führt, wenn Unterstützung für mehrere Plattformen und Betriebssysteme gewünscht wird.\autocite[vgl.][]{Maj}
Progressive Web Apps revolutionieren diesen Ansatz.\\
Als Webseite, die jegliche Eigenschaften einer nativen Applikation aufweist, kann sie auf dem Endgerät über den Webbrowser installiert werden. Diese Eigenschaften beinhalten Installierbarkeit, Offlinefähigkeit oder Push-Benachrichtigungen, selbst wenn die Anwendung geschlossen wurde. Die progressive Web App stellt daher eine Mischung aus responsiver Webseite und nativen App dar.
% Vorgehensweise
Diese plattformübergreifende Web Apps werden auf der Grundlage von HTML5, CSS3 und JavaScript erstellt. Das HTTP-Protokoll zur Kommunikation zwischen Webclient und Webserver stellt dabei die grundlegende Bedingung zur Verwendung einer PWA dar.
Die PWA enthält sogenannte \textit{ServiceWorker}, welche Funktionalitäten wie Offlinefähigkeit, Caching als auch Push-Benachrichtigungen ermöglichen. Bei einem ServiceWorker handelt es sich um einen in JavaScript implementierten Proxy, der zwischen der Anwendung und dem Server geschaltet wird.\autocite[vgl.][]{Hume.2018}
% Vorteile
Die Verwendung einer PWA bietet sowohl für Entwickler als auch für den Anwender viele Vorteile. Durch den Mobile-First-Ansatz wird garantiert, dass die Web Apps problemlos auf mobilen Geräten innerhalb des Browsers funktionieren. Somit ist die User Experience wie in er nativen App und bietet viele bekannte Vorteile während ist die Entwicklung in vielen Fällen um ein Vielfaches schneller, einfacher und kostengünstiger ist als die einer nativen App.

