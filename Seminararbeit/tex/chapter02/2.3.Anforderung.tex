% !TEX root =  master.tex
\section{Anforderungsanalyse}
% Funktioale Anforderungen und nicht funktionale Anforderungen
Eine zentrale Fragestellung bei der Konzeption von progressiven Web Apps ist die nach den Zielen der Anwendung. Welche Aufgaben soll der Benutzer erledigen können und welche qualitativen sowie technischen Rahmenbedingungen sollen dabei berücksichtigt werden? Nach diesen Fragen wurde eine Liste von funktionalen und nicht-funktionalen Anforderungen erstellt, welche sich auf die erste Version der Xpire-App, das Minimal Viable Product (MVP), beziehen:

\subsection{Funktionale Anforderungen}
Die funktionalen Anforderungen beschreiben die Funktionalitäten, welche die Anwendung bereitstellen soll:
\begin{itemize}[noitemsep]
	\item[F1] \textbf{Produkte hinterlegen:} Der Kunde soll die Möglichkeit haben, gekaufte Produkte per Barcode-Scan oder manuell hinterlegen zu können.
	\item[F2] \textbf{Produktinformationen hinterlegen:} Der Benutzer hat die Möglichkeit folgende Informationen zu den Produkten zu hinterlegen: Titel, Anzahl, Einkaufsdatum und Gültigkeitsdatum.
	\item[F3] \textbf{Produkte verwalten:} Die Produkte sollen dem Benutzer in übersichtlicher Weise angezeigt werden. Der Nutzer hat die Möglichkeit ein Produkt zu löschen oder die hinterlegten Informationen zu den Produkten jederzeit zu verändern.
	\item[F4] \textbf{Bild hinterlegen:} Benutzer sollen die Möglichkeit haben, pro Produkt ein Bild hochladen zu können.
	\item[F5] \textbf{Benachrichtigungen erhalten:} Der Benutzer sollen eine Push-Benachrichtigung erhalten, wenn ein Produkt kurz vor dem Ablauf seiner Haltbarkeit steht.
\end{itemize}

\subsection{Nicht-funktionale Anforderungen}
Die nicht-funktionale Anforderungen betreffen die Umstände, unter denen die geforderten Funktionalitäten zu erbringen sind:

% Welche User-Entwickler Dokumentation wurde bereitgestellt, wo und warum? (Giithub Readme)

\begin{itemize}[noitemsep]
	\item[NF1] \textbf{Usability-Ziele:} Die Usability (dt.: Gebrauchstauglichkeit) ist definiert als das Ausmaß, in dem ein Produkt in einem bestimmten Anwendungskontext genutzt werden kann, um bestimmte Ziele mit geringem Aufwand und hoher Zufriedenheit zu erreichen.\autocite[vgl.][S.3 ff.]{Balzert.2009} Basierend auf den Ergebnissen der Benutzerprofilanalyse wurde
	der Fokus bei der erstellten progressiven Web App auf die Erreichung folgender Kriterien gelegt:
	\begin{itemize}[noitemsep]
	\item[NF1.1] \textbf{Verständliche Informationsstruktur:}Eine klare Informationsstruktur schafft die Voraussetzung für die intuitive Bedienbarkeit des Systems und kann damit grundlegend zu einer positiven Benutzererfahrung beitragen.\autocite[vgl.][S.106 ff.]{Moser.2012} Aus diesem Grund sollen bereitgestellte Informationen möglichst logisch strukturiert und leicht verständlich sein. Grundsätzlich sollten dem Anwender nur die im aktuellen Kontext benötigten Informationen präsentiert und Unwichtiges weggelassen werden.
	\item[NF1.2] \textbf{Durchdachtes Navigationskonzept:}Eine gut gestaltete Navigation erleichtert dem Benutzer die Orientierung und helfen ihm die folgenden Fragen zu beantworten: Wo befinde ich mich, wie bin ich hier hergekommen und wohin kann ich von hier aus gehen?\autocite[vgl.][S.116 ff.]{Moser.2012}
	\end{itemize}
	\item[NF1] \textbf{Technische Rahmenbedingungen:}Aus den technischen Randbedingungen wird in der Regel die Wahl der einzusetzenden Technologien getroffen. Diese sind:
	\begin{itemize}[noitemsep]
		\item[NF1.1] \textbf{Physikalische Nutzungsumgebung:} Einsatzort der PWA kann überall sein. Um eine optimale Lesbarkeit zu garantieren, sollte daher auf die richtige Verwendung von Farbkontrasten und Schriftarten geachtet werden.
		\item[NF1.2] \textbf{Hardware:} Als Endgerät kann sowohl ein Smartphone, als auch ein Tablet oder ein PC verwendet werden. Die PWA muss daher so konzpipiert sein, dass sie auf die Eigenschaften des jeweils benutzten Endgeräts optimal reagieren kann.
	\end{itemize}
\end{itemize}
 