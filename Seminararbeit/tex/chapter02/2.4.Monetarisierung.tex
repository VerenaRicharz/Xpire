% !TEX root =  master.tex
\section{Monetarisierungsstrategie}
Grundsätzlich gibt es zahlreiche Möglichkeiten, um mit einer App Geld zu verdienen. Wichtig ist, im Vorfeld zu überlegen, wie die App auf dem Markt positioniert werden soll und welche Monetarisierungsstrategie am besten passt. Zu den drei bekanntesten Möglichkeiten zählen:
\begin{itemize}[noitemsep]
	\item \textbf{Paid-App:} Dies ist nicht die einfachste Möglichkeit, um Geld mit einer App zu verdienen. Statt die App kostenlos im Google Play Store oder im Apple Store anzubieten, wird die App kostenpflichtig angeboten, sodass dem Nutzer einmalige Kosten beim Download der App entstehen.
	\item \textbf{In-App-Werbung:} Das Schalten von Werbung innerhalb der App ist eine beliebte Methode um den Umsatz anzukurbeln. Die Möglichkeiten sind sehr vielfältig und reichen von klassischen Werbebannern bis hin zu boomender Videowerbung.
	\item \textbf{Freemium:} Bei dieser Monetarisierungsstrategie ist der Download der App kostenlos, der User hat dann die Möglichkeit durch In-App-Käufe oder einem Abo neue Inhalte, neue Funktionen oder die Laufzeit einer Vollversion zu verlängern. Dafür fallen dem User dann zusätzliche Kosten an.
\end{itemize}
Neben den genannten Beispielen gibt es noch viele weitere Möglichkeiten zur Monetarisierung einer App. Wir haben die verschiedenen Szenarien genauer beleuchtet und sind zu folgender Auswertung gekommen:\\
Eine Paid-App kommt für uns nicht in Frage, da wird die Xpire-App ausschließlich kostenlos anbieten möchten. Unser Ziel ist es, möglichst viele User zu gewinnen um so für mehr Nachhaltigkeit und einen verantwortungsvolleren Umgang mit Lebensmitteln zu sorgen. Nur wenige Benutzer sind bereit, für eine App Geld auszugeben und die Gefahr, das andere Anbieter eine ähnliche, kostenlose Alternative anbieten, ist hoch. Daher scheidet diese Strategie aus.
Das Schalten von Werbung sehen wir als guten Kompromiss, um die App für den User kostenlos zu gestalten und trotzdem Umsatz zu generieren. Falls wir zukünftig Werbung schalten werden, darf diese jedoch weder unseriös, aufdringlich oder nervig wirken und muss gegen Bezahlung deaktiviert werden können.
Das Anbieten von zusätzlichen In-App-Käufe oder das Abschließen eines Abos um auf neue Inhalte oder Funktionen zugreifen zu können, halten wir in unserem Fall jedoch für wenig sinnvoll.\\
Es sind zusätzliche Funktionen wie das Aufzeigen von passenden Angeboten der umliegenden Supermärkte oder das Vorschlagen von Rezepten basierend auf den hinterlegten Produkten geplant. Diese zusätzlichen Funktionen sollen aber keine kostenpflichtigen Features darstellen und ebenfalls kostenlos angeboten werden. Vielmehr halten wir die Kooperation mit verschiedenen Rezeptseiten wie Chefkoch.de oder restegourmet.de als auch die Kooperation mit diversen Supermärkten für sinnvoll. Darüber hinaus besteht die Möglichkeit die App teilweise durch Spenden zu finanzieren und kostenlose Infrastruktur bei verschiedenen Hosting-Anbietern anzufragen, denn einige Anbieter stellen gemeinnützigen Vereinen kostenlosen Speicherplatz zu Verfügung.





